%  LaTeX support: latex@mdpi.com 
%  For support, please attach all files needed for compiling as well as the log file, and specify your operating system, LaTeX version, and LaTeX editor.

%=================================================================
\documentclass[journal,ijms,article,submit,pdftex,moreauthors]{Definitions/mdpi} 
%\documentclass[preprints,article,submit,pdftex,moreauthors]{Definitions/mdpi} 
% For posting an early version of this manuscript as a preprint, you may use "preprints" as the journal. Changing "submit" to "accept" before posting will remove line numbers.

% Below journals will use APA reference format:
% admsci, aieduc, behavsci, businesses, econometrics, economies, education, ejihpe, famsci, games, humans, ijcs, ijfs, journalmedia, jrfm, languages, psycholint, publications, tourismhosp, youth

% Below journals will use Chicago reference format:
% arts, genealogy, histories, humanities, jintelligence, laws, literature, religions, risks, socsci

%---------
% article
%---------
% The default type of manuscript is "article", but can be replaced by: 
% abstract, addendum, article, benchmark, book, bookreview, briefcommunication, briefreport, casereport, changes, clinicopathologicalchallenge, comment, commentary, communication, conceptpaper, conferenceproceedings, correction, conferencereport, creative, datadescriptor, discussion, entry, expressionofconcern, extendedabstract, editorial, essay, erratum, fieldguide, hypothesis, interestingimages, letter, meetingreport, monograph, newbookreceived, obituary, opinion, proceedingpaper, projectreport, reply, retraction, review, perspective, protocol, shortnote, studyprotocol, supfile, systematicreview, technicalnote, viewpoint, guidelines, registeredreport, tutorial,  giantsinurology, urologyaroundtheworld
% supfile = supplementary materials

%----------
% submit
%----------
% The class option "submit" will be changed to "accept" by the Editorial Office when the paper is accepted. This will only make changes to the frontpage (e.g., the logo of the journal will get visible), the headings, and the copyright information. Also, line numbering will be removed. Journal info and pagination for accepted papers will also be assigned by the Editorial Office.

%------------------
% moreauthors
%------------------
% If there is only one author the class option oneauthor should be used. Otherwise use the class option moreauthors.

%---------
% pdftex
%---------
% The option pdftex is for use with pdfLaTeX. Remove "pdftex" for (1) compiling with LaTeX & dvi2pdf (if eps figures are used) or for (2) compiling with XeLaTeX.

%=================================================================
% MDPI internal commands - do not modify
\firstpage{1} 
\makeatletter 
\setcounter{page}{\@firstpage} 
\makeatother
\pubvolume{1}
\issuenum{1}
\articlenumber{0}
\pubyear{2025}
\copyrightyear{2025}
%\externaleditor{Firstname Lastname} % More than 1 editor, please add `` and '' before the last editor name
\datereceived{ } 
\daterevised{ } % Comment out if no revised date
\dateaccepted{ } 
\datepublished{ } 
%\datecorrected{} % For corrected papers: "Corrected: XXX" date in the original paper.
%\dateretracted{} % For retracted papers: "Retracted: XXX" date in the original paper.
\hreflink{https://doi.org/} % If needed use \linebreak
%\doinum{}
%\pdfoutput=1 % Uncommented for upload to arXiv.org
%\CorrStatement{yes}  % For updates
%\longauthorlist{yes} % For many authors that exceed the left citation part

% Full title of the paper (Capitalized)
\Title{Benchmarking Summarization Methods for Scientific Abstracts: From Classical Models to LLMs}

% MDPI internal command: Title for citation in the left column
\TitleCitation{Title}

% Author Orchid ID: enter ID or remove command
\newcommand{\orcidauthorA}{0000-0000-0000-000X} % Add \orcidA{} behind the author's name
\newcommand{\orcidauthorB}{0009-0007-7272-5034} % Add \orcidB{} behind the author's name; PP
\newcommand{\orcidauthorC}{0000-0003-2087-5691} % Add \orcidB{} behind the author's name; EB
\newcommand{\orcidauthorD}{0000-0001-6853-8542} % Add \orcidB{} behind the author's name; ML

% Authors, for the paper (add full first names)
\Author{Fabio Baumgärtel $^{1}$, Enrico Bono $^{1,2}$\orcidB{}, Paul Perco $^{1,3}$\orcidC{} and Matthias Ley $^{1,2}$\orcidD{}*}

% MDPI internal command: Authors, for metadata in PDF
\AuthorNames{Fabio Baumgärtel, Enrico Bono, Paul Perco and Matthias Ley}

% Affiliations / Addresses (Add [1] after \address if there is only one affiliation.)
\address{%
$^{1}$ \quad Delta4 GmbH, Vienna, Austria\\
$^{2}$ \quad Division of Pediatric Nephrology and Gastroenterology, Department of Pediatrics and Adolescent Medicine, Comprehensive Center for Pediatrics, Medical University Vienna, Vienna, Austria\\
$^{3}$ \quad Department of Internal Medicine IV, Medical University Innsbruck, Innsbruck, Austria}

% Contact information of the corresponding author
\corres{Correspondence: matthias.ley@delta4.ai}

% Current address and/or shared authorship
%\firstnote{Current address: Affiliation.}  
% Current address should not be the same as any items in the Affiliation section.

%\secondnote{These authors contributed equally to this work.}
% The commands \thirdnote{} till \eighthnote{} are available for further notes.

%\simplesumm{} % Simple summary

%\conference{} % An extended version of a conference paper

% Abstract (Do not insert blank lines, i.e. \\) 
\abstract{A single paragraph of about 200 words maximum. For research articles, abstracts should give a pertinent overview of the work. We strongly encourage authors to use the following style of structured abstracts, but without headings: (1) Background: place the question addressed in a broad context and highlight the purpose of the study; (2) Methods: describe briefly the main methods or treatments applied; (3) Results: summarize the article's main findings; (4) Conclusions: indicate the main conclusions or interpretations. The abstract should be an objective representation of the article, it must not contain results which are not presented and substantiated in the main text and should not exaggerate the main conclusions.}

% Keywords
\keyword{benchmarking; natural language processing; text summarization; largue language models} 

% The fields PACS, MSC, and JEL may be left empty or commented out if not applicable
%\PACS{J0101}
%\MSC{}
%\JEL{}

%%%%%%%%%%%%%%%%%%%%%%%%%%%%%%%%%%%%%%%%%%
% Only for the journal Diversity
%\LSID{\url{http://}}

%%%%%%%%%%%%%%%%%%%%%%%%%%%%%%%%%%%%%%%%%%
% Only for the journal Applied Sciences
%\featuredapplication{Authors are encouraged to provide a concise description of the specific application or a potential application of the work. This section is not mandatory.}
%%%%%%%%%%%%%%%%%%%%%%%%%%%%%%%%%%%%%%%%%%

%%%%%%%%%%%%%%%%%%%%%%%%%%%%%%%%%%%%%%%%%%
% Only for the journal Data
%\dataset{DOI number or link to the deposited data set if the data set is published separately. If the data set shall be published as a supplement to this paper, this field will be filled by the journal editors. In this case, please submit the data set as a supplement.}
%\datasetlicense{License under which the data set is made available (CC0, CC-BY, CC-BY-SA, CC-BY-NC, etc.)}

%%%%%%%%%%%%%%%%%%%%%%%%%%%%%%%%%%%%%%%%%%
% Only for the journal BioTech, Fishes, Neuroimaging and Toxins
%\keycontribution{The breakthroughs or highlights of the manuscript. Authors can write one or two sentences to describe the most important part of the paper.}

%%%%%%%%%%%%%%%%%%%%%%%%%%%%%%%%%%%%%%%%%%
% Only for the journal Encyclopedia
%\encyclopediadef{For entry manuscripts only: please provide a brief overview of the entry title instead of an abstract.}

%%%%%%%%%%%%%%%%%%%%%%%%%%%%%%%%%%%%%%%%%%
% Only for the journal Advances in Respiratory Medicine, Future, Sensors and Smart Cities
%\addhighlights{yes}
%\renewcommand{\addhighlights}{%
%
%\noindent This is an obligatory section in ``Advances in Respiratory Medicine'', ``Future'', ``Sensors'' and ``Smart Cities”, whose goal is to increase the discoverability and readability of the article via search engines and other scholars. Highlights should not be a copy of the abstract, but a simple text allowing the reader to quickly and simplified find out what the article is about and what can be cited from it. Each of these parts should be devoted up to 2~bullet points.\vspace{3pt}\\
%\textbf{What are the main findings?}
% \begin{itemize}[labelsep=2.5mm,topsep=-3pt]
% \item First bullet.
% \item Second bullet.
% \end{itemize}\vspace{3pt}
%\textbf{What is the implication of the main finding?}
% \begin{itemize}[labelsep=2.5mm,topsep=-3pt]
% \item First bullet.
% \item Second bullet.
% \end{itemize}
%}

%%%%%%%%%%%%%%%%%%%%%%%%%%%%%%%%%%%%%%%%%%
\begin{document}

%%%%%%%%%%%%%%%%%%%%%%%%%%%%%%%%%%%%%%%%%%

\section{Introduction}

The exponential growth of scientific literature has created a demand for text summarization methods to support scientists in efficiently prioritizing papers, extracting relevant information, and interpreting complex findings. \ac{ATS} methods have evolved from statistical approaches to deep learning-based models, becoming increasingly sophisticated and reliable at capturing essential parts from complex research articles. \ac{ATS} methods have been previously evaluated and described \cite{zhang2025comprehensivesurveyprocessorientedautomatic,zhang2024systematicsurveytextsummarization}, but few are tailored for scientific literature summarization \cite{ROHIL2022100058,xie2023surveybiomedicaltextsummarization}. 

\subsection{Statistical and Encoder-Decoder Approaches}

The pre-neural era of text summarization was mainly characterized by extractive approaches, where in an unsupervised way, summaries were generated by using word or concept frequencies to identify relevant sentences. The first word-frequency based approaches were discussed by Luhn \cite{Luhn1958TheAC}, who presented a method based on the assumption that recurrent words in a text are likely more important. Later, Edmundson \cite{10.1145/321510.321519} introduced concepts such as cue words, title words, and sentence position to further enhance the automatic summarization process. The concept of \ac{TF-IDF} was later adopted \cite{article} and applied to text summarization by representing sentences as term-weight vectors that down-weight frequently occurring terms with low context specificity in an entire corpus of documents, while promoting rarer terms that at the same time are more context-specific. Word-frequency based approaches have been extensively used in scientific text summarization, being at the basis of more sophisticated strategies \cite{10.1145/1183614.1183701}. Finally, graph-based statistical methods, in which sentences are represented as nodes and their relationships as edges weighted by similarity measures (e.g. cosine similarity of \ac{TF-IDF} vectors), allow for the identification of the most central information based on the documents global structure rather than local word counts. A widely used approach in the biomedical domain is TextRank, which constructs a sentence graph to then apply the PageRank algorithm to compute importance sentence importance scores, and generates the summary by selecting the top-ranked sentences \cite{mihalcea-tarau-2004-textrank}. 

With the advent of \ac{Seq2seq} frameworks, summarization shifted toward neural approaches that paraphrase and condense text using Encoder-Decoder architectures, originally implemented with \acp{RNN}, \ac{LSTM} networks, and \acp{GRU} \cite{afzal2020, almasoud2022}. The introduction of self-attention mechanisms replaced \ac{Seq2seq} frameworks by processing sequences in parallel rather than sequentially, enabling the capture of complex linguistic patterns and long-range contextual relationships \cite{vaswani2023attentionneed}. This innovation laid the foundation for transformer architectures that quickly gained popularity in performing a wide range of \ac{NLP} tasks, including text summarization. One of the earliest and most influential transformer-based models, \ac{BERT} \cite{devlin2019bertpretrainingdeepbidirectional}, was widely adopted in domain specific tasks owing to its possibility to be fine-tuned by adding a task-specific output layer. Inspired by \ac{BERT}'s architecture, several abstractive summarization models emerged, including \ac{BART} - a denoising autoencoder for pretraining \ac{Seq2seq} models \cite{DBLP:journals/corr/abs-1910-13461} that can be trained or fine-tuned on scientific literature \cite{yuan-etal-2022-biobart, abinaya2024}. The \ac{T5} model was introduced as a unified text-to-text framework for a broad spectrum of \ac{NLP} tasks due to its high flexibility with no need for architectural changes \cite{JMLR:v21:20-074}. \ac{PEGASUS}, was specifically proposed for abstractive summarization \cite{zhang2020pegasuspretrainingextractedgapsentences} and has been adapted for scientific text with domain-specific variants including “google/pegasus-pubmed” and “google/bigbird-pegasus-large-pubmed”. \ac{RoBERTa} is an optimized version of \ac{BERT} trained on a bigger corpus of text, which led to the creation of Longformer \cite{Beltagy2020Longformer}, a transformer-based architecture that can handle longer texts for text-to-text generation, with “allenai/led-base-16384” and “led-large-16384-arxiv” as notable examples \cite{steblianko2024}.

\subsection{Language Models}

Despite these advances, the field of \ac{ATS} quickly moved towards decoder-only architectures which are at the basis of \acp{LLM}, able to capture semantic relations with higher flexibility and specificity. \acp{LLM} can be classified as (i) general-purpose models, which leverage their broad domain knowledge across diverse \ac{NLP} tasks, (ii) reasoning-oriented models, characterized by logical text understanding through iterative chain-of-thought processing and instruction tuning \cite{plaat2025multistepreasoninglargelanguage}, and (iii) domain-specific models, tailored for specialized tasks or scientific domains. Several families of \acp{LLM} have been developed, including the \ac{GPT} series developed by OpenAI (GPT-1 \cite{Radford2018ImprovingLU} through GPT-5) and open-source variants like GPT:OSS, all pre-trained on large-scale text corpora through self-supervised learning. Similarly, Anthropic's Claude Models are built on transformer architecture and trained through a Constitutional AI approach \cite{bai2022constitutionalaiharmlessnessai}. This family also includes a series of reasoning models such as Sonnet-4 and Opus-4. Meta's \ac{Llama} family \cite{grattafiori2024llama3herdmodels}, with LLaMA 3.1 as the most capable open-source model available to date, includes domain-specific adaptations such as OpenBioLLM-LLaMA-3 \cite{OpenBioLLMs}, a biomedical variant trained on a large corpus of high-quality biomedical data, and MedLLaMA-2 \cite{touvron2023llama2openfoundation}, a medical \ac{LM} based on LLaMA 2 architecture. Google developed a series of lightweight models including the Gemma series \cite{gemmateam2025gemma3technicalreport}, with Gemma3 as its latest and most powerful reasoning model. Microsoft introduced the Phi series \cite{abdin2024phi3technicalreporthighly, abdin2024phi4technicalreport}, which comprises Phi-4-reasoning and Phi-4-mini-reasoning, alongside BioGPT \cite{10.1093/bib/bbac409}, a domain-specific model built on the \ac{GPT} architecture and fine-tuned for biomedical applications. IBM released the Granite series \cite{mishra2024granitecodemodelsfamily}, with Granite 4.0 as its reasoning-capable variant. Mistral AI developed the Mistral family \cite{jiang2023mistral7b}, including Magistral as its first reasoning model \cite{mistralai2025magistral}, and Biomistral \cite{labrak2024biomistral}, an open-source variant pretrained on PubMed Central data for biomedical text processing. Alibaba Cloud introduced the Qwen 3 series \cite{qwen3technicalreport} as an open-source LLM family, which inspired SciLitLLM \cite{li2025scilitllmadaptllmsscientific}, a specialized model for scientific literature understanding based on Qwen2.5 and trained through \ac{CPT} and \ac{SFT} on scientific literature \cite{li2025scilitllmadaptllmsscientific}. DeepSeek has developed \ac{RL}-driven reasoning models that achieve performance comparable to state-of-the-art closed-source models while requiring only a fraction of their training costs \cite{wang2025reviewdeepseekmodelskey}. Lastly, Apertus \cite{swissai2025apertus} represents Switzerland’s first large-scale open, multilingual \ac{LM} with a fully documented and openly accessible development process. 

To the best of our knowledge, no comprehensive benchmarking of text summarization models on biomedical literature has been reported to date. This study addresses this gap by systematically evaluating 62 summarization models, ranging from word-frequency methods to state-of-the-art \acp{LLM}, using a curated dataset of 1,000 biomedical abstracts and corresponding highlights sections as reference summaries for benchmarking. By identifying the strengths and limitations of each approach, we provide actionable insights for selecting appropriate summarization tools to accelerate knowledge discovery in biomedical sciences.  

%%%%%%%%%%%%%%%%%%%%%%%%%%%%%%%%%%%%%%%%%%

\section{Materials and Methods}

\subsection{Gold-Standard Dataset}

To establish a reliable benchmark for automatic summarization, we assembled a gold-standard dataset of 1,000 biomedical articles drawn from a diverse set of peer-reviewed journals hosted on \textit{ScienceDirect} and \textit{Cell Press}. These journals were selected because, in addition to their focus on molecular and biomedical sciences, they provide a standardized \textit{Highlights} section. This section provides concise bullet points that capture the main findings of each article. These served as the reference summaries in our evaluation, while the corresponding abstracts were used as the input texts for the summarization.

Articles were collected systematically across a variety of journals to ensure coverage of different fields within molecular sciences such as drug discovery, genomics, proteomics, biotechnology, and biochemistry. We selected 50 articles from each of the 20 journals, bringing the dataset to 1,000 in total. The distribution of articles across journals is summarized in Table~\ref{tab:goldstandard}.

\begin{table}[H]
\centering
\caption{Overview of journals and number of articles included in the gold-standard dataset.\label{tab:goldstandard}}
\begin{tabularx}{\textwidth}{llc}
\toprule
\textbf{Publisher} & \textbf{Journal} & \textbf{Articles included} \\
\midrule
ScienceDirect & Drug Discovery Today & 50 \\
ScienceDirect & Journal of Molecular Biology & 50 \\
ScienceDirect & FEBS Letters & 50 \\
ScienceDirect & Journal of Biotechnology & 50 \\
ScienceDirect & Gene & 50 \\
ScienceDirect & Genomics & 50 \\
ScienceDirect & Journal of Proteomics & 50 \\
ScienceDirect & The International Journal of Biochemistry \& Cell Biology & 50 \\
ScienceDirect & Cytokine & 50 \\
ScienceDirect & Developmental Cell & 50 \\
Cell & Cell & 50 \\
Cell & Cancer Cell & 50 \\
Cell & Cell Chemical Biology & 50 \\
Cell & Cell Genomics & 50 \\
Cell & Cell Host \& Microbe & 50 \\
Cell & Cell Metabolism & 50 \\
Cell & Cell Reports & 50 \\
Cell & Cell Reports Medicine & 50 \\
Cell & Cell Stem Cell & 50 \\
Cell & Cell Systems & 50 \\
\bottomrule
\end{tabularx}
\end{table}

This setup provides standardized pairs of abstracts and reference summaries that can be directly used for evaluating automatic summarization methods.

\subsection{Summarization Methods}

We evaluated 50 summarization methods, ranging from simples frequency-based algorithms to state-of-the-art large language models (LLMs). By having this extensive coverage of methods, we were able to compare established techniques with the latest transformer-based mdodels under identical conditions.

The models were grouped into five categories:

\begin{enumerate}
	\item Traditional methods: As a foundation for comparison, we included two traditional extractive methods: a simple frequency-based approach and TextRank. These methods provide a simple baseline to compare the more complex approaches with.
	\item Encoder-Decoder models: We included a set of pretrained encoder-decoder models, which are available through the HuggingFace library: BART (base and large), T5 (base and large), mT5, a variety of PEGASUS models, and LED. These models are often applied for abstractive summarization and represent well-established neural systems within our benchmark.
	\item General-purpose LLMs: <TBD, still have to decide on the final groupings>
	\item Reasoning-oriented LLMs: <TBD, still have to decide on the final groupings>
	\item Specialized LLMs: To assess whether domain adapatation improves summarization quality, we included large language models additionally trained on medical/biomedical data, such as MedLlama2 and other specialized models.
\end{enumerate}

The complete list of models included in each category is shown in Table~\ref{tab:models}.

\begin{table}[H]
\caption{Overview of summarization methods/models evaluated in this study, organized by category.\label{tab:models}}
\begin{tabularx}{\textwidth}{lX}
\toprule
\textbf{Group} & \textbf{Methods/Models} \\
\midrule
Traditional methods & textrank; frequency \\
\midrule
HuggingFace transformers & bart-large-cnn; bart-base; t5-base; t5-large; mT5\_multilingual\_XLSum; pegasus-xsum; pegasus-large; pegasus-cnn\_dailymail; led\_large\_16384\_arxiv\_summarization \\
\midrule
Specialized LLMs & medllama2:7b; openbiollm-llama-3:8b\_q8\_0 \\
\midrule
General-purpose LLMs & gemma3:1b; gemma3:4b; gemma3:12b; granite3.3:2b; granite3.3:8b; llama3.1:8b; llama3.2:1b; llama3.2:3b; mistral:7b; mistral-nemo:12b; mistral-small3.2:24b; gemma3-tools:4b; phi3:3.8b; phi4:14b; qwen3:4b; qwen3:8b; gpt-3.5-turbo; gpt-4.1; gpt-4.1-mini; gpt-4o; gpt-4o-mini; claude-3.5-haiku; claude-sonnet-4; mistral-medium; mistral-small; mistral-large; gpt-5-nano; gpt-5-mini \\
\midrule
Reasoning-oriented LLMs & deepseek-r1:1.5b; deepseek-r1:7b; deepseek-r1:8b; deepseek-r1:14b; gpt-oss:20b; claude-opus-4; gpt-5; magistral-medium; claude-opus-4-1; \\
\bottomrule
\end{tabularx}
\end{table}

With this selection we covered models of different sizes and release periods, ensuring that both widely adopted systems and recent architectures were represented. Extraordinarily large models were not considered as they are impractical for typical summarization pipelines and fall outside the scope of our benchmarking goals.

These 50 diverse models were all tasked with generating summaries for each of the 1,000 abstracts in the dataset, resulting in 50,000 generated summaries up for evaluation.

\subsection{Evaluation Metrics}

As there is no single metric that can fully reflect summary quality, especially in the biomedical field where both coverage of key information and factual correctness are critical, we used a multitude of metrics grouped into five categories: traditional surface-level metrics, embedding-based similarity metrics, content coverage metrics, factuality metrics and performance-related measures that reflect the feasibility of using the methods in actual real-world applications. By combining all these metrics into one final overall score, we end up with a balanced benchmark value that reflects both summary quality and practical usability.

\subsubsection{Surface-level Metrics}

This group consists of metrics that compare the generated summaries with the reference summaries mainly at the word or phrase level. While they do not capture meaning beyond surface overlap, they remain common metrics in summarization research and provide a simple foundation for evaluation. We used three ROUGE variants (ROUGE-1, ROUGE-2, ROUGE-L), BLEU and METEOR. ROUGE-1 and ROUGE-2 measure how many unigrams (single words) or bigrams (word pairs) from the reference appear in the generated output, while ROUGE-L identifies the longest sequence of words shared between the two. BLEU calculates how many n-grams in the output also occur in the reference, but it emphasizes precision rather than recall and applies a brevity penalty to counteract the tendency toward overly short summaries. METEOR extends n-gram matching by also considering word stems and synonyms, which makes it more tolerant to variations in wording. Together, these metrics offer a simple but transparent point of reference.

\subsubsection{Embedding-based Similarity Metrics}

To capture similarity beyond surface-level word overlap, we also included embedding-based metrics built on pretrained transformer models. These methods generate vector representations of the texts, which allows them to capture similarity in meaning rather than just word overlap. Specifically, we employed RoBERTa and DeBERTa, two transformer-based models that have shown strong performance on a variety of natural language processing (NLP) tasks. In the case of summarization evaluation, they can be used to judge whether two summaries capture the same content even if phrased differently.

\subsubsection{Content Coverage Metrics}

<TBD, maybe also add to Embedding-based Similarity Metrics?>

\subsubsection{Factuality Metrics}

For factual consistency we used AlignScore, a metric designed to assess whether the statements in a generated summary are supported by the source text. In contrast to the other metrics, we used AlignScore in a way where it does not compare the output to the reference summary but instead aligns it directly with the abstract, as factual correctness can only be judged relative to the input text itself and not against a condensed reference. … By adding AlignScore we have a metric that is sensitive to errors and halucinations…

\subsubsection{Performance Metrics}

\subsection{Benchmarking Framework}
<TBD>

\subsection{Computational Resources}
<TBD>


%%%%%%%%%%%%%%%%%%%%%%%%%%%%%%%%%%%%%%%%%%

\section{Results}

Our benchmark results offer a comparative view of summarization performance across all evaluated models. We first present overall rankings, followed by comparisons between different model groups. Additionally, we examine results on individual metrics, runtime performance, and correlations between the evaluation metrics used.

\subsection{Overall Model Performance}

Based on the performance outcomes shown in Figure~\ref{fig:rank_heatmap}, models from the Mistral family occupied the top positions of the ranking, achieving strong performance across the majority of surface-level metrics (ROUGE-1, ROUGE-2, ROUGE-L, METEOR, BLEU) and embedding-based measures (RoBERTa, DeBERTa, all-mpnet-base-v2, AlignScore). According to the performance rank, mistral-medium-2505 ranks first, followed by mistral-small-2506, mistral-small-3.2:24b, and mistral-large-2411. 

The lowest-ranked models include pegasus-xsum, pegasus-pubmed and bigbird-pegasus-large-pubmed from the Pegasus family, with the latter being the worst-performing model. Domain-specific models such as OpenBioLLM-Llama3-8B, biogpt, and multilingual\_XLSum, show poor performance across all the metrics.

Among the 10 top ranked models, five are general-purpose LLMs, three are general-purpose SLMs, and two are reasoning-oriented LLMs. A similar trend is present by looking at the top half of the ranking (positions 10 to 32), except from the presence of a single domain-specific LLM ranked at 20 (SciLitLLM1.5-14B). In contrast, in the lower half of the ranking, where models start to perform poorly across most metrics, the majority are reasoning-oriented SLMs, general-purpose EDMs, domain-specific EDMs, and traditional models. The best and worst models by category are reported in Table~\ref{tab:bestworst}, while those by model family are reported in Table~\ref{tab:bestworstfamily}. 

% Define consistent column widths
\newlength{\colA}\setlength{\colA}{3.2cm}  % Category / Family
\newlength{\colB}\setlength{\colB}{3.5cm}  % Best Model
\newlength{\colC}\setlength{\colC}{0.7cm}  % Rank
\newlength{\colD}\setlength{\colD}{3.7cm}  % Worst Model
\newlength{\colE}\setlength{\colE}{0.7cm}  % Rank

%-------------------- TABLE 4 --------------------
\begin{table}[H]
\caption{Overview of the best- and worst-performing models by category. Only categories with at least 3 models are reported.\label{tab:bestworst}}
\centering
\begin{tabularx}{\textwidth}{p{\colA} p{\colB} p{\colC} p{\colD} p{\colE}}
\toprule
\textbf{Category} & \textbf{Best Model} & \textbf{Rank} & \textbf{Worst Model} & \textbf{Rank} \\
\midrule
General-purpose EDMs & \texttt{T5-large} & 48 & \texttt{Pegasus-large} & 55 \\
Domain-specific EDMs & \texttt{led\_large\_16384\_ arxiv\_summarization} & 46 & \texttt{bigbird-pegasus- large-pubmed} & 62 \\
General-purpose SLMs & \texttt{GPT-4o-mini} & 6 & \texttt{Phi3:3.8b} & 45 \\
General-purpose LLMs & \texttt{Mistral-medium-2505} & 1 & \texttt{Mistral-nemo:12b} & 30 \\
Reasoning-oriented SLMs & \texttt{qwen3:8b} & 35 & \texttt{Deepseek-r1:8b} & 50 \\
Reasoning-oriented LLMs & \texttt{GPT-5-nano} & 7 & \texttt{GPT-5} & 41 \\
Domain-specific SLMs & \texttt{SciLitLLM1.5-7B} & 37 & \texttt{OpenBioLLM-Llama3-8B} & 61 \\
\bottomrule
\end{tabularx}
\end{table}

%-------------------- TABLE 5 --------------------
\begin{table}[H]
\caption{Overview of the best- and worst-performing models by family. Only families with at least 3 models are reported.\label{tab:bestworstfamily}}
\centering
\begin{tabularx}{\textwidth}{p{\colA} p{\colB} p{\colC} p{\colD} p{\colE}}
\toprule
\textbf{Family} & \textbf{Best Model} & \textbf{Rank} & \textbf{Worst Model} & \textbf{Rank} \\
\midrule
Pegasus & \texttt{pegasus-cnn\_ dailymail} & 49 & \texttt{bigbird-pegasus- large-pubmed} & 62 \\
T5 & \texttt{T5-large} & 48 & \texttt{mT5\_multi- lingual\_XLSum} & 60 \\
Qwen & \texttt{SciLitLLM1.5-14B} & 20 & \texttt{Qwen3:4b} & 43 \\
Gemma & \texttt{Gemma3-tools:4b} & 10 & \texttt{Gemma3:270M} & 42 \\
Granite & \texttt{Granite3.3:8b} & 9 & \texttt{Granite3.3:2b} & 28 \\
LLaMA & \texttt{Llama3.2:3b} & 26 & \texttt{OpenBioLLM-Llama3-8B} & 61 \\
Mistral & \texttt{Mistral-medium-2505} & 1 & \texttt{BioMistral-7B} & 38 \\
Phi & \texttt{Phi4:14b} & 23 & \texttt{Phi3:3.8b} & 45 \\
DeepSeek & \texttt{Deepseek-r1:14b} & 34 & \texttt{Deepseek-r1:8b} & 50 \\
GPT & \texttt{GPT-4o} & 5 & \texttt{BioGPT} & 59 \\
Claude & \texttt{Claude-sonnet-4} & 8 & \texttt{Claude-opus-4-1} & 33 \\
\bottomrule
\end{tabularx}
\end{table}

\begin{figure}[H]
\centering
\includegraphics[width=1.00\textwidth]{Visualizations/rank_heatmap.png}
\caption{Overview of the performance of all evaluated models across all surface-level and embedding-based metrics. Each row corresponds to one model, and each column to a specific metric, with lower ranks indicating better performance. The figure displays each model’s family (e.g., GPT, DeepSeek, Gemma, Granite), and category (e.g., encoder–decoder, general-purpose SLMs, reasoning-oriented LLMs). Models are sorted by their weighted average rank across metrics (Performance\_Rank), where lower ranks indicate better performance.\label{fig:rank_heatmap}}
\end{figure}

\subsection{Group Comparisons}

Figure~\ref{fig:group_bar_chart} summarizes the average performance of the nine model categories based on the overall metric mean score. General-purpose LLMs achieved the highest mean score (0.527), followed by general-purpose SLMs (0.519) and reasoning-oriented LLMs (0.515). Traditional extractive models, general-purpose EDMs, and domain-specific EDMs performed considerably lower, with mean scores of 0.451, 0.471, and 0.410, respectively. The domain-specific SLMs showed weak overall performance (0.439), whereas the domain-specific LLMs achieved a higher score (0.513; single model).

\begin{figure}[H]
\centering
\includegraphics[width=1.0\textwidth]{Visualizations/group_bar_chart.png}
\caption{Average metric mean score across the nine model categories. The figure highlights clear performance differences between categories, with general-purpose LLMs performing best overall, followed by general-purpose SLMs and reasoning-oriented LLMs. Traditional models, EDMs, and domain-specific SLMs achieved notably lower scores.\label{fig:group_bar_chart}}
\end{figure}

\subsubsection{SLMs vs. LLMs}

To further analyze differences between small and large language models, we compared the performance of SLMs and LLMs within both the general-purpose and reasoning-oriented groups (Figure~\ref{fig:slm_llm_comparison}). In both categories, LLMs achieved higher overall metric mean scores (0.527 vs 0.519 and 0.515 vs. 0.495) and generally performed better on surface-level and embedding-based metrics. Compliance with word-length bounds favored LLMs in the general-purpose group but SLMs in the reasoning-oriented group. The comparison for domain-specific models is omitted, as this category includes only a single LLM, preventing meaningful comparison.

\begin{figure}[H]
\centering
\subfloat[\centering General-purpose SLMs vs.\ LLMs.]{%
    \includegraphics[width=0.49\textwidth]{Visualizations/generalpurpose_comparison.png}
}
\hfill
\subfloat[\centering Reasoning-oriented SLMs vs.\ LLMs.]{%
    \includegraphics[width=0.49\textwidth]{Visualizations/reasoning_comparison.png}
}
\caption{(\textbf{a}) Comparison between general-purpose SLMs and LLMs across key evaluation metrics. (\textbf{b}) Comparison between reasoning-oriented SLMs and LLMs. In both groups, LLMs achieved slightly higher overall metric mean scores, while SLMs occasionally performed better on individual metrics. \label{fig:slm_llm_comparison}}
\end{figure}

\subsubsection{General-purpose Models vs. Reasoning-oriented Models}
    
Figure~\ref{fig:llm_group_comparison}a compares the two largest and most competitive groups (general-purpose and reasoning-oriented models) across multiple evaluation aspects, including both SLMs and LLMs. General-purpose models achieved slightly higher scores across surface-level metrics, embedding-based metrics, execution time, compliance with word-length bounds, and overall metric mean score. The largest difference was observed in execution time, where general-purpose models reached a score of 0.968 compared to 0.930 for reasoning-oriented models. Figure~\ref{fig:llm_group_comparison}b provides a detailed view of these runtime differences. Smaller but consistent advantages were also seen in surface-level metrics, embedding-based metrics, and compliance with word-length bounds.

\begin{figure}[H]
\centering
\subfloat[\centering General-purpose vs.\ reasoning-oriented models across key evaluation aspects.]{%
    \includegraphics[width=0.49\textwidth]{Visualizations/llm_comparison.png}
}
\hfill
\subfloat[\centering Execution time distribution for the same two groups.]{%
    \includegraphics[width=0.49\textwidth]{Visualizations/grouped_execution_time_distribution.png}
}
\caption{(\textbf{a}) Comparison between general-purpose and reasoning-oriented models across key evaluation metrics. General-purpose models achieved higher scores across all categories, including surface-level and embedding-based metrics, execution time, compliance with word-length bounds, and overall Metric Mean Score. (\textbf{b}) Distribution of execution times for the same groups, showing that general-purpose models produced summaries more efficiently and with lower variability.\label{fig:llm_group_comparison}}
\end{figure}

\subsection{Metric Correlations}

To examine how the different evaluation metrics relate to each other, we computed pairwise Pearson correlation coefficients across all models (Figure~\ref{fig:metric_correlation}).

Strong positive correlations were observed among the surface-level metrics (ROUGE-1, ROUGE-2, ROUGE-L, METEOR, and BLEU). ROUGE variants showed almost identical behavior ($\rho > 0.9$), while BLEU and METEOR demonstrated slightly weaker but still substantial alignment with ROUGE measures.

Most embedding-based metrics (RoBERTa, DeBERTa, and all-mpnet-base-v2) showed very high internal consistency ($\rho > 0.8$), reflecting their shared focus on semantic similarity beyond surface-level overlap. When compared with surface-level metrics, correlations were moderate to strong ($\rho \approx 0.7$--$1.0$), indicating that both categories capture related but not identical dimensions of summary quality.

AlignScore correlated moderately with the other metrics ($\rho \approx 0.4$--$0.7$), which can be attributed to its different point of reference, as it compares generated summaries directly with source abstracts instead of the reference summaries like other metrics.

Overall, these relationships demonstrate that the various metrics are broadly consistent while providing complementary perspectives. This supports the use of an aggregated ``Metrics Mean Score'' as a balanced indicator of overall summarization performance.

\begin{figure}[H]
\centering
\includegraphics[width=0.75\textwidth]{Visualizations/metric_correlation.png}
\caption{Correlation matrix of evaluation metrics. Each cell in the matrix represents the Pearson correlation coefficient ($\rho$) between two metrics based on their mean scores across all models. Surface-level and embedding-based metrics show strong internal consistency, while AlignScore exhibits lower correlations due to its distinct focus on factual consistency with source abstracts.\label{fig:metric_correlation}}
\end{figure}

\subsection{Qualitative Analysis: Case Study}

To illustrate how concept coverage differs between models beyond aggregate metrics, we examine two summaries of a biomedical research article [doi:10.1016/j.cell.2025.05.026] against publisher-provided highlights. The source article, titled \textit{``A percolation phase transition controls complement protein coating of surfaces''}, includes four key highlights:  
(H1) The complement protein network has a switch-like response when attacking surfaces;  
(H2) Complement "decides" to coat surfaces if surface protein spacing is below a threshold;  
(H3) Complement’s threshold decision-making arises from a percolation phase transition;  
(H4) Complexity science shows how complement makes discrete decisions attacking surfaces.  

The generated summaries are as follows:

\begin{itemize}
    \item \textbf{mistral\_medium-2505:} "This study reveals that the complement system, a network of blood proteins, exhibits a sharp activation threshold based on the surface density of attachment points on materials. This threshold behavior, observed across scales from nanoparticles to macroscale pathologies, arises from a percolation-type phase transition in a minimal subnetwork of complement proteins. The findings suggest that this critical transition acts as a switch, determining whether the complement system will interact with and initiate inflammation against a material."
    \item \textbf{SciLitLLM1.5-14B:} "A sharp threshold for complement activation upon varying a fundamental material parameter, the surface density of potential complement attachment points, was experimentally demonstrated and computationally explained."
\end{itemize}

Table~\ref{tab:concept_coverage} summarizes the concept-level coverage and overall performance for both models. While both models capture the surface density threshold (H2), the lower-performing model omits the mechanistic explanation (H3) and only partially conveys the behavioral characteristics (H1, H4), resulting in substantially lower concept coverage and semantic alignment. The overall performance scores, derived using the weighted metric aggregation introduced in Section~\ref{subsec:ranks_calculation}, further reflect this difference, with \textbf{mistral\_medium-2505} achieving 0.619 compared to 0.560 for \textbf{SciLitLLM1.5-14B}.

\begin{table}[H]
\centering
\caption{Concept coverage analysis of model-generated summaries. Symbols: \checkmark = fully covered; $\sim$ = partially covered; $\times$ = not covered. Overall performance scores are derived using the weighted metric aggregation described in Section~\ref{subsec:ranks_calculation}.}
\label{tab:concept_coverage}
\begin{tabularx}{0.9\textwidth}{lcc}
\toprule
\textbf{Reference Concept} & \textbf{mistral\_medium-2505} & \textbf{SciLitLLM1.5-14B} \\
\midrule
H1: Switch-like response & \checkmark & $\sim$ \\
H2: Surface density threshold & \checkmark & \checkmark \\
H3: Percolation phase transition & \checkmark & $\times$ \\
H4: Discrete decision-making & \checkmark & $\sim$ \\
\midrule
Coverage Score & 4.0 / 4.0 & 1.5 / 4.0 \\
Semantic Similarity\textsuperscript{a} & 0.946 & 0.731 \\
Overall Performance Score & 0.619 & 0.560 \\
\bottomrule
\end{tabularx}

\smallskip
\raggedright
\textsuperscript{a}\textit{Cosine similarity to highlights (all-mpnet-base-v2).}
\end{table}

%%%%%%%%%%%%%%%%%%%%%%%%%%%%%%%%%%%%%%%%%%

\section{Discussion}

\subsection{Overview of Main Findings}

The benchmarking analysis revealed clear performance differences between the evaluated summarization approaches. Overall, general-purpose large language models (LLMs) achieved the highest summarization quality across all surface-level and embedding-based metrics, followed closely by general-purpose small language models (SLMs) and reasoning-oriented LLMs. In contrast, domain-specific scientific/biomedical models, encoder–decoder architectures such as T5 and PEGASUS, and traditional extractive methods like TextRank all reached noticeably lower performance levels. These results highlight the clear progression from extractive and encoder–decoder approaches toward transformer-based models, while also showing that domain-specific fine-tuning alone does not necessarily lead to improved summarization quality.

\subsection{Model Group Comparisons}

To understand the causes of the observed performance differences, the models were compared by architecture, size, and domain focus. This analysis examines how model scale, reasoning ability, and domain specialization influence summarization quality in biomedical texts. The next sections discuss these aspects in detail by comparing large and small language models, general-purpose and scientific/biomedical models, and general-purpose and reasoning-oriented models.

\subsubsection{Large vs. Small Language Models (LLMs vs. SLMs)}

-discuss relationship between model size and summarization performance. https://arxiv.org/html/2501.05465v1

\subsubsection{General-purpose vs. Scientific/Biomedical Models}

-discuss why general-purpose models outperformed scientific/biomedical ones. https://arxiv.org/abs/2408.13833

\subsubsection{General-purpose vs Reasoning-oriented Models}

-discuss why reasoning-oriented models did not surpass general-purpose ones in summarization (primarily needs semantic compression and factual grounding rather than multi-step logical reasoning). https://arxiv.org/abs/2504.08120

\subsection{Evaluation and Metric Considerations}

-reflect how different evaluation metrics capture complementary aspects of summary quality. explain distinction between surface-level and embedding-based metrics. discuss observed correlations

\subsection{Limitations and Future Work}

-state main limitations (focus on single summarization task: abstract -> highlights, rapid evolution, absence of human quality evaluation)

\subsection{Practical Implications and Applications}

-emphasize how the results can guide model selection in biomedical NLP. highlight tradeoff between accuracy and efficiency.
-conclude with short statement that general-purpose LLMs currently provide the most robust option for scientific summarization.

%%%%%%%%%%%%%%%%%%%%%%%%%%%%%%%%%%%%%%%%%%

\input{Sections/conclusion}

%%%%%%%%%%%%%%%%%%%%%%%%%%%%%%%%%%%%%%%%%%

\section{Results}

This section may be divided by subheadings. It should provide a concise and precise description of the experimental results, their interpretation as well as the experimental conclusions that can be drawn.
\subsection{Subsection}
\subsubsection{Subsubsection}

Bulleted lists look like this:
\begin{itemize}
\item	First bullet;
\item	Second bullet;
\item	Third bullet.
\end{itemize}

Numbered lists can be added as follows:
\begin{enumerate}
\item	First item; 
\item	Second item;
\item	Third item.
\end{enumerate}

The text continues here.

\subsection{Figures, Tables and Schemes}

All figures and tables should be cited in the main text as Figure~\ref{fig1}, Table~\ref{tab1}, etc.

\begin{figure}[H]
%\isPreprints{\centering}{} % Only used for preprints
\includegraphics[width=4.0 cm]{Definitions/logo-mdpi}
\caption{This is a figure. Schemes follow the same formatting.\label{fig1}}
\end{figure}   
\unskip

\begin{table}[H] 
%\small % Change table font size
\caption{This is a table caption. Tables should be placed in the main text near to the first time they are~cited.\label{tab1}}
%\isPreprints{\centering}{} % Only used for preprints
\begin{tabularx}{\textwidth}{CCC}
\toprule
\textbf{Title 1}	& \textbf{Title 2}	& \textbf{Title 3}\\
\midrule
Entry 1		& Data			& Data\\
Entry 2		& Data			& Data \textsuperscript{1}\\
\bottomrule
\end{tabularx}

\noindent{\footnotesize{\textsuperscript{1} Tables may have a footer.}}
\end{table}

The text continues here (Figure~\ref{fig2} and Table~\ref{tab2}).

% Example of a figure that spans the whole page width and with subfigures. The same concept works for tables, too.
\begin{figure}[H]
%\isPreprints{} % If the paper is ``preprints'', please uncomment this parenthesis.
\subfloat[\centering]{\includegraphics[width=7.0cm]{Definitions/logo-mdpi}}
%\hfill
\subfloat[\centering]{\includegraphics[width=7.0cm]{Definitions/logo-mdpi}}\\
\subfloat[\centering]{\includegraphics[width=7.0cm]{Definitions/logo-mdpi}}
%\hfill
\subfloat[\centering]{\includegraphics[width=7.0cm]{Definitions/logo-mdpi}}
%\isPreprints{} % If the paper is ``preprints'', please uncomment this parenthesis.
\caption{This is a wide figure. Schemes follow the same formatting. If there are multiple panels, they should be listed as: (\textbf{a}) Description of what is contained in the first panel. (\textbf{b}) Description of what is contained in the second panel. (\textbf{c}) Description of what is contained in the third panel. (\textbf{d}) Description of what is contained in the fourth panel. Figures should be placed in the main text near to the first time they are cited. A caption on a single line should be centered.\label{fig2}}
\end{figure} 

\begin{table}[H]
\caption{This is a wide table.\label{tab2}}
%\isPreprints{\centering} % If the paper is ``preprints'', please uncomment this parenthesis.
%\isPreprints{\begin{tabularx}{\textwidth}{CCCC}} % If the paper is ``preprints'', please uncomment this parenthesis.
			\toprule
			\textbf{Title 1}	& \textbf{Title 2}	& \textbf{Title 3}     & \textbf{Title 4}\\
			\midrule
\multirow[m]{3}{*}{Entry 1 *}	& Data			& Data			& Data\\
			  	                   & Data			& Data			& Data\\
			             	      & Data			& Data			& Data\\
                   \midrule
\multirow[m]{3}{*}{Entry 2}    & Data			& Data			& Data\\
			  	                  & Data			& Data			& Data\\
			             	     & Data			& Data			& Data\\
			\bottomrule
		\end{tabularx}
%		\isPreprints{} % If the paper is ``preprints'', please uncomment this parenthesis.
	\noindent{\footnotesize{* Tables may have a footer.}}
\end{table}

%\begin{listing}[H]
%\caption{Title of the listing}
%\rule{\columnwidth}{1pt}
%\raggedright Text of the listing. In font size footnotesize, small, or normalsize. Preferred format: left aligned and single spaced. Preferred border format: top border line and bottom border line.
%\rule{\columnwidth}{1pt}
%\end{listing}

Text.

Text.

\subsection{Formatting of Mathematical Components}

This is the example 1 of equation:
\begin{linenomath}
\begin{equation}
a = 1,
\end{equation}
\end{linenomath}
the text following an equation need not be a new paragraph. Please punctuate equations as regular text.
%% If the documentclass option "submit" is chosen, please insert a blank line before and after any math environment (equation and eqnarray environments). This ensures correct linenumbering. The blank line should be removed when the documentclass option is changed to "accept" because the text following an equation should not be a new paragraph.

This is the example 2 of equation:
%\isPreprints{} % If the paper is ``preprints'', please uncomment this parenthesis.
\begin{equation}
a = b + c + d + e + f + g + h + i + j + k + l + m + n + o + p + q + r + s + t + u + v + w + x + y + z
\end{equation}
%\isPreprints{} % If the paper is ``preprints'', please uncomment this parenthesis.

%% Example of a page in landscape format (with table and table footnote).
%\startlandscape
%\begin{table}[H] %% Table in wide page
%%\isPreprints{\centering}{} % This command is only used for ``preprints''.
%\caption{This is a very wide table.\label{tab3}}
%	\begin{tabularx}{\textwidth}{CCCC}
%		\toprule
%		\textbf{Title 1}	& \textbf{Title 2}	& \textbf{Title 3}	& \textbf{Title 4}\\
%		\midrule
%		Entry 1		& Data			& Data			& This cell has some longer content that runs over two lines.\\
%		Entry 2		& Data			& Data			& Data\textsuperscript{1}\\
%		\bottomrule
%	\end{tabularx}
%%\isPreprints{}{% This command is only used for ``preprints''.
%	\begin{adjustwidth}{+\extralength}{0cm}
%%} % If the paper is ``preprints'', please uncomment this parenthesis.
%		\noindent\footnotesize{\textsuperscript{1} This is a table footnote.}
%%\isPreprints{}{% This command is only used for ``preprints''.
%	\end{adjustwidth}
%%} % If the paper is ``preprints'', please uncomment this parenthesis.
%\end{table}
%\finishlandscape


Please punctuate equations as regular text. Theorem-type environments (including propositions, lemmas, corollaries etc.) can be formatted as follows:
%% Example of a theorem:
\begin{Theorem}
Example text of a theorem.
\end{Theorem}

The text continues here. Proofs must be formatted as follows:

%% Example of a proof:
\begin{proof}[Proof of Theorem 1]
Text of the proof. Note that the phrase ``of Theorem 1'' is optional if it is clear which theorem is being referred to.
\end{proof}
The text continues here.

%%%%%%%%%%%%%%%%%%%%%%%%%%%%%%%%%%%%%%%%%%
\section{Discussion}

Authors should discuss the results and how they can be interpreted from the perspective of previous studies and of the working hypotheses. The findings and their implications should be discussed in the broadest context possible. Future research directions may also be highlighted.

%%%%%%%%%%%%%%%%%%%%%%%%%%%%%%%%%%%%%%%%%%
\section{Conclusions}

This section is not mandatory, but can be added to the manuscript if the discussion is unusually long or complex.

%%%%%%%%%%%%%%%%%%%%%%%%%%%%%%%%%%%%%%%%%%
\section{Patents}

This section is not mandatory, but may be added if there are patents resulting from the work reported in this manuscript.

%%%%%%%%%%%%%%%%%%%%%%%%%%%%%%%%%%%%%%%%%%
\vspace{6pt} 

%%%%%%%%%%%%%%%%%%%%%%%%%%%%%%%%%%%%%%%%%%
%% optional
%\supplementary{The following supporting information can be downloaded at:  \linksupplementary{s1}, Figure S1: title; Table S1: title; Video S1: title.}

% Only for journal Methods and Protocols:
% If you wish to submit a video article, please do so with any other supplementary material.
% \supplementary{The following supporting information can be downloaded at: \linksupplementary{s1}, Figure S1: title; Table S1: title; Video S1: title. A supporting video article is available at doi: link.}

% Only used for preprtints:
% \supplementary{The following supporting information can be downloaded at the website of this paper posted on \href{https://www.preprints.org/}{Preprints.org}.}

% Only for journal Hardware:
% If you wish to submit a video article, please do so with any other supplementary material.
% \supplementary{The following supporting information can be downloaded at: \linksupplementary{s1}, Figure S1: title; Table S1: title; Video S1: title.\vspace{6pt}\\
%\begin{tabularx}{\textwidth}{lll}
%\toprule
%\textbf{Name} & \textbf{Type} & \textbf{Description} \\
%\midrule
%S1 & Python script (.py) & Script of python source code used in XX \\
%S2 & Text (.txt) & Script of modelling code used to make Figure X \\
%S3 & Text (.txt) & Raw data from experiment X \\
%S4 & Video (.mp4) & Video demonstrating the hardware in use \\
%... & ... & ... \\
%\bottomrule
%\end{tabularx}
%}

%%%%%%%%%%%%%%%%%%%%%%%%%%%%%%%%%%%%%%%%%%
\authorcontributions{For research articles with several authors, a short paragraph specifying their individual contributions must be provided. The following statements should be used ``Conceptualization, X.X. and Y.Y.; methodology, X.X.; software, X.X.; validation, X.X., Y.Y. and Z.Z.; formal analysis, X.X.; investigation, X.X.; resources, X.X.; data curation, X.X.; writing---original draft preparation, X.X.; writing---review and editing, X.X.; visualization, X.X.; supervision, X.X.; project administration, X.X.; funding acquisition, Y.Y. All authors have read and agreed to the published version of the manuscript.'', please turn to the  \href{http://img.mdpi.org/data/contributor-role-instruction.pdf}{CRediT taxonomy} for the term explanation. Authorship must be limited to those who have contributed substantially to the work~reported.}

\funding{Please add: ``This research received no external funding'' or ``This research was funded by NAME OF FUNDER grant number XXX.'' and  and ``The APC was funded by XXX''. Check carefully that the details given are accurate and use the standard spelling of funding agency names at \url{https://search.crossref.org/funding}, any errors may affect your future funding.}

\institutionalreview{In this section, you should add the Institutional Review Board Statement and approval number, if relevant to your study. You might choose to exclude this statement if the study did not require ethical approval. Please note that the Editorial Office might ask you for further information. Please add “The study was conducted in accordance with the Declaration of Helsinki, and approved by the Institutional Review Board (or Ethics Committee) of NAME OF INSTITUTE (protocol code XXX and date of approval).” for studies involving humans. OR “The animal study protocol was approved by the Institutional Review Board (or Ethics Committee) of NAME OF INSTITUTE (protocol code XXX and date of approval).” for studies involving animals. OR “Ethical review and approval were waived for this study due to REASON (please provide a detailed justification).” OR “Not applicable” for studies not involving humans or animals.}

\informedconsent{Any research article describing a study involving humans should contain this statement. Please add ``Informed consent was obtained from all subjects involved in the study.'' OR ``Patient consent was waived due to REASON (please provide a detailed justification).'' OR ``Not applicable'' for studies not involving humans. You might also choose to exclude this statement if the study did not involve humans.

Written informed consent for publication must be obtained from participating patients who can be identified (including by the patients themselves). Please state ``Written informed consent has been obtained from the patient(s) to publish this paper'' if applicable.}

\dataavailability{We encourage all authors of articles published in MDPI journals to share their research data. In this section, please provide details regarding where data supporting reported results can be found, including links to publicly archived datasets analyzed or generated during the study. Where no new data were created, or where data is unavailable due to privacy or ethical restrictions, a statement is still required. Suggested Data Availability Statements are available in section ``MDPI Research Data Policies'' at \url{https://www.mdpi.com/ethics}.} 

% Only for journal Drones
%\durcstatement{Current research is limited to the [please insert a specific academic field, e.g., XXX], which is beneficial [share benefits and/or primary use] and does not pose a threat to public health or national security. Authors acknowledge the dual-use potential of the research involving xxx and confirm that all necessary precautions have been taken to prevent potential misuse. As an ethical responsibility, authors strictly adhere to relevant national and international laws about DURC. Authors advocate for responsible deployment, ethical considerations, regulatory compliance, and transparent reporting to mitigate misuse risks and foster beneficial outcomes.}

% Only for journal Nursing Reports
%\publicinvolvement{Please describe how the public (patients, consumers, carers) were involved in the research. Consider reporting against the GRIPP2 (Guidance for Reporting Involvement of Patients and the Public) checklist. If the public were not involved in any aspect of the research add: ``No public involvement in any aspect of this research''.}
%
%% Only for journal Nursing Reports
%\guidelinesstandards{Please add a statement indicating which reporting guideline was used when drafting the report. For example, ``This manuscript was drafted against the XXX (the full name of reporting guidelines and citation) for XXX (type of research) research''. A complete list of reporting guidelines can be accessed via the equator network: \url{https://www.equator-network.org/}.}
%
%% Only for journal Nursing Reports
%\useofartificialintelligence{Please describe in detail any and all uses of artificial intelligence (AI) or AI-assisted tools used in the preparation of the manuscript. This may include, but is not limited to, language translation, language editing and grammar, or generating text. Alternatively, please state that “AI or AI-assisted tools were not used in drafting any aspect of this manuscript”.}

\acknowledgments{In this section you can acknowledge any support given which is not covered by the author contribution or funding sections. This may include administrative and technical support, or donations in kind (e.g., materials used for experiments). Where GenAI has been used for purposes such as generating text, data, or graphics, or for study design, data collection, analysis, or interpretation of data, please add “During the preparation of this manuscript/study, the author(s) used [tool name, version information] for the purposes of [description of use]. The authors have reviewed and edited the output and take full responsibility for the content of this publication.”}

\conflictsofinterest{Declare conflicts of interest or state ``The authors declare no conflicts of interest.'' Authors must identify and declare any personal circumstances or interest that may be perceived as inappropriately influencing the representation or interpretation of reported research results. Any role of the funders in the design of the study; in the collection, analyses or interpretation of data; in the writing of the manuscript; or in the decision to publish the results must be declared in this section. If there is no role, please state ``The funders had no role in the design of the study; in the collection, analyses, or interpretation of data; in the writing of the manuscript; or in the decision to publish the results''.} 

%%%%%%%%%%%%%%%%%%%%%%%%%%%%%%%%%%%%%%%%%%
%% Optional

%% Only for journal Encyclopedia
%\entrylink{The Link to this entry published on the encyclopedia platform.}

\abbreviations{Abbreviations}{
The following abbreviations are used in this manuscript:
\\

\noindent 
\begin{tabular}{@{}ll}
MDPI & Multidisciplinary Digital Publishing Institute\\
DOAJ & Directory of open access journals\\
TLA & Three letter acronym\\
LD & Linear dichroism
\end{tabular}
}

%%%%%%%%%%%%%%%%%%%%%%%%%%%%%%%%%%%%%%%%%%
%% Optional
\appendixtitles{no} % Leave argument "no" if all appendix headings stay EMPTY (then no dot is printed after "Appendix A"). If the appendix sections contain a heading then change the argument to "yes".
\appendixstart
\appendix
\section[\appendixname~\thesection]{}
\subsection[\appendixname~\thesubsection]{}
The appendix is an optional section that can contain details and data supplemental to the main text---for example, explanations of experimental details that would disrupt the flow of the main text but nonetheless remain crucial to understanding and reproducing the research shown; figures of replicates for experiments of which representative data are shown in the main text can be added here if brief, or as Supplementary Data. Mathematical proofs of results not central to the paper can be added as an appendix.

\begin{table}[H] 
\caption{This is a table caption.\label{tab5}}
%\newcolumntype{C}{>{\centering\arraybackslash}X}
\begin{tabularx}{\textwidth}{CCC}
\toprule
\textbf{Title 1}	& \textbf{Title 2}	& \textbf{Title 3}\\
\midrule
Entry 1		& Data			& Data\\
Entry 2		& Data			& Data\\
\bottomrule
\end{tabularx}
\end{table}

\section[\appendixname~\thesection]{}
All appendix sections must be cited in the main text. In the appendices, Figures, Tables, etc. should be labeled, starting with ``A''---e.g., Figure A1, Figure A2, etc.

%%%%%%%%%%%%%%%%%%%%%%%%%%%%%%%%%%%%%%%%%%
%\isPreprints{} % If the paper is ``preprints'', please uncomment this parenthesis.
%\printendnotes[custom] % Un-comment to print a list of endnotes

\reftitle{References}

% Please provide either the correct journal abbreviation (e.g. according to the “List of Title Word Abbreviations” http://www.issn.org/services/online-services/access-to-the-ltwa/) or the full name of the journal.
% Citations and References in Supplementary files are permitted provided that they also appear in the reference list here. 

%=====================================
% References, variant A: external bibliography
%=====================================
\bibliography{bibliography}

%=====================================
% References, variant B: internal bibliography
%=====================================

% If authors have biography, please use the format below
%\section*{Short Biography of Authors}
%\bio
%{\raisebox{-0.35cm}{\includegraphics[width=3.5cm,height=5.3cm,clip,keepaspectratio]{Definitions/author1.pdf}}}
%{\textbf{Firstname Lastname} Biography of first author}
%
%\bio
%{\raisebox{-0.35cm}{\includegraphics[width=3.5cm,height=5.3cm,clip,keepaspectratio]{Definitions/author2.jpg}}}
%{\textbf{Firstname Lastname} Biography of second author}

% For the MDPI journals use author-date citation, please follow the formatting guidelines on http://www.mdpi.com/authors/references
% To cite two works by the same author: \citeauthor{ref-journal-1a} (\citeyear{ref-journal-1a}, \citeyear{ref-journal-1b}). This produces: Whittaker (1967, 1975)
% To cite two works by the same author with specific pages: \citeauthor{ref-journal-3a} (\citeyear{ref-journal-3a}, p. 328; \citeyear{ref-journal-3b}, p.475). This produces: Wong (1999, p. 328; 2000, p. 475)

%%%%%%%%%%%%%%%%%%%%%%%%%%%%%%%%%%%%%%%%%%
%% for journal Sci
%\reviewreports{\\
%Reviewer 1 comments and authors’ response\\
%Reviewer 2 comments and authors’ response\\
%Reviewer 3 comments and authors’ response
%}
%%%%%%%%%%%%%%%%%%%%%%%%%%%%%%%%%%%%%%%%%%
\PublishersNote{}
%\isPreprints{} % If the paper is ``preprints'', please uncomment this parenthesis.
\end{document}

